\documentclass[]{article}

\usepackage{amsmath}
\usepackage{amsfonts}
\usepackage{amsthm}
\usepackage{amssymb}
\usepackage{cleveref}
\usepackage{courier}
\usepackage{natbib}
\usepackage{subcaption}

\DeclareMathOperator{\var}{var}
\DeclareMathOperator*{\argmax}{arg\,max}

\setlength{\parindent}{15pt}
\newtheorem{theorem}{Theorem}
\newtheorem{lemma}{Lemma}
\newtheorem{corollary}{Corollary}
\newtheorem{definition}{Definition}
\theoremstyle{remark}
\newtheorem*{remark}{Remark}

\newcommand{\EE}{\mathbb{E}}
\newcommand{\PP}{\mathbb{P}}
\newcommand{\RR}{\mathbb{R}}
\newcommand*{\vertbar}{\rule[-1ex]{0.5pt}{2.5ex}}
\newcommand*{\horzbar}{\rule[.5ex]{2.5ex}{0.5pt}}
\DeclareMathOperator{\tr}{tr}
\DeclareMathOperator{\diag}{diag}

\begin{document}

\begin{itemize}

\item Kenneth Hung
\item Response to reviews
\item Reference: Rank verification for exponential families

\end{itemize}

\section{Response to Reviewer 1}

\begin{itemize}

\item {\em Then they are offering a test based on the first two order statistics. If that is the case they should say so.}

We adopted this suggestion.

\item {\em It is said that we would be asking different questions if Cruz had led the race? What are they? Wouldn't we still be asking if Cruz was really winning?}

We clarified what we refer to as different questions in the paper.

\item {\em Why would we stop with a failure to reject? Could we not obtain later orders? What if we wanted to know who was last?}

This is to stay in line with BasicStop \citep{Fithian:2015uj}, also known as the partition test, where we reject a sequence of null hypothesis and halt whenever we can no longer reject.

In the case where we want to know who was last, we can take $-X$ as the observations and $-\theta$ as the parameter. Applying our procedure will thus infer about the last. This is mentioned in Section 1.2 and a similar idea was included in \citet{Stefansson:1988wj} as well.

\item {\em In primary voting, the second place finisher is not necessarily the second choice. If the choice of Trump was removed it is not clear that Cruz would move to first.}

We are not inferring that Cruz (the first running-up in polls) to be the second choice. We are testing that Cruz has the second largest support. This is not clarified in Section 1.2 where we define what ``best'', ``second best'', ``winner'' and ``runner-up'' mean.

\item {\em What is the definition of a ”valid” test being used here?}

A valid test should control the Type I error rate (in the case of just testing for the best), and FWER (in the case of verifying multiple ranks). This has been made more explicit in the paper.

\item {\em On line 17 from the bottom of page 4, the sentence beginning with ``Berger (1980) ...'' does not seem to make any sense.}

Thanks for catching that. It is now fixed.

\item {\em What can be said about using all the data? Suppose there were just 3 populations and you decided that the first two were different. Would you feel the same way about the difference of the first two if you next found that the 2nd and 3rd were not different?}

We would not feel any different about the population. We are performing a test to establish the winner as the best by comparing the top two order statistics. Even the third population is closer to the second population, the test remains statistically valid, and we can establish the same conclusion.

In a case with a more complex joint distribution, changing $X_3$ would alter the test, as the test comparing $X_1$ and $X_2$ conditions on $X_3$. The test procedure remains the same.

\item {\em This is not a step-down procedure in the sense of multiple testing. There the order used to test hypotheses depends on the individual p-values.}

That is a point quite valid. We agree that calling this a step-down procedure can be misleading. We have changed this to ``stepwise procedure''. However, if there is a wider known name for this type of ``reject until we can no longer reject'' procedure, we would gladly change the name again.

\item {\em Again suppose there were just a few populations and you wanted to know about all the ranks. As the authors point out, the method works beginning either with the maximum or minimum. However they need not give the same result. How would their performance differ?}

Since it is a stepwise procedure, it will ``get stuck'' in cases with parameters clustered together. The performance of starting from either end depends largely on the distribution of the parameters. If a few parameters are clustered on the lower end, the starting the test from the minimum will result in acceptance of the unadjusted pairwise test and no further ranks will be verified. On the other hand, if the tests are performed from the maximum, then most of the rank can be verified, save for the few clustered in the lower end. We believe in most applications there is one end that holds more significance (e.g.\ drug trial).

\end{itemize}

\section{Response to Reviewer 2}

\begin{itemize}

\item {\em The main reason for pointing out this reference is its technique for constructing a confidence set is by what is called Partitioning, a technique which has turned out to be fundamentally useful in multiple comparisons.}

We included this reference and credited this reference for its test on comparison with sample best under location family with monotone likelihood ratio.

\end{itemize}

\bibliographystyle{plainnat}
\bibliography{papers}

\end{document}